%%%%%%%%%%%%%%%%%%%%%%%%%%%%%%%%%%%%%%%%%
% Short Sectioned Assignment
% LaTeX Template
% Version 1.0 (5/5/12)
%
% This template has been downloaded from:
% http://www.LaTeXTemplates.com
%
% Original author:
% Frits Wenneker (http://www.howtotex.com)
%
% License:
% CC BY-NC-SA 3.0 (http://creativecommons.org/licenses/by-nc-sa/3.0/)
%
%%%%%%%%%%%%%%%%%%%%%%%%%%%%%%%%%%%%%%%%%

%----------------------------------------------------------------------------------------
%	PACKAGES AND OTHER DOCUMENT CONFIGURATIONS
%----------------------------------------------------------------------------------------

\documentclass[paper=a4, fontsize=11pt]{scrartcl} % A4 paper and 11pt font size

\usepackage[T1]{fontenc} % Use 8-bit encoding that has 256 glyphs
% \usepackage{fourier} % Use the Adobe Utopia font for the document - comment this line to return to the LaTeX default
\usepackage[english]{babel} % English language/hyphenation
\usepackage{amsmath,amsfonts,amsthm} % Math packages
\usepackage{siunitx}
\usepackage{mathtools} %for delimiters
\usepackage{braket}

\DeclarePairedDelimiter\abs{\lvert}{\rvert}%
\DeclarePairedDelimiter\norm{\lVert}{\rVert}%
% Swap the definition of \abs* and \norm*, so that \abs
% and \norm resizes the size of the brackets, and the
% starred version does not.
\makeatletter
\let\oldabs\abs
\def\abs{\@ifstar{\oldabs}{\oldabs*}}
%
\let\oldnorm\norm
\def\norm{\@ifstar{\oldnorm}{\oldnorm*}}
\makeatother

\sisetup{
    %output-decimal-marker={,}% just uncomment if you want to use comma as the decimal marker!
}

\usepackage{sectsty} % Allows customizing section commands
\allsectionsfont{\centering \normalfont\scshape} % Make all sections centered, the default font and small caps

\usepackage{fancyhdr} % Custom headers and footers
\pagestyle{fancyplain} % Makes all pages in the document conform to the custom headers and footers
\fancyhead{} % No page header - if you want one, create it in the same way as the footers below
\fancyfoot[L]{} % Empty left footer
\fancyfoot[C]{} % Empty center footer
\fancyfoot[R]{\thepage} % Page numbering for right footer
\renewcommand{\headrulewidth}{0pt} % Remove header underlines
\renewcommand{\footrulewidth}{0pt} % Remove footer underlines
\setlength{\headheight}{13.6pt} % Customize the height of the header

\numberwithin{equation}{section} % Number equations within sections (i.e. 1.1, 1.2, 2.1, 2.2 instead of 1, 2, 3, 4)
\numberwithin{figure}{section} % Number figures within sections (i.e. 1.1, 1.2, 2.1, 2.2 instead of 1, 2, 3, 4)
\numberwithin{table}{section} % Number tables within sections (i.e. 1.1, 1.2, 2.1, 2.2 instead of 1, 2, 3, 4)

\setlength\parindent{0pt} % Removes all indentation from paragraphs - comment this line for an assignment with lots of text

%----------------------------------------------------------------------------------------
%	TITLE SECTION
%----------------------------------------------------------------------------------------

\newcommand{\horrule}[1]{\rule{\linewidth}{#1}} % Create horizontal rule command with 1 argument of height

\title{
\normalfont \normalsize
\textsc{The Open University} \\ [25pt] % Your university, school and/or department name(s)
\horrule{0.5pt} \\[0.4cm] % Thin top horizontal rule
\huge SXP390 Tutor Marked Assignment 03 - Part 4 \\ % The assignment title
\horrule{2pt} \\[0.5cm] % Thick bottom horizontal rule
}

\author{William Ockmore} % Your name

\date{\normalsize\today} % Today's date or a custom date

\begin{document}

\maketitle % Print the title

%----------------------------------------------------------------------------------------
%	PROBLEM 1
%----------------------------------------------------------------------------------------

\section{The BB84 and B92 prepare and measure protocols}

\subsection{BB84}

The BB84 protocol was the first proposed QKD protocol, put forward as the name implies in
1984 by Bennet and Brassard in their paper. %TODO insert reference%

\subsubsection{The protocol}
In BB84, Alice uses a single photon source to transmit a set of states to Bob,
encoded in the polarization of the photons. Both Alice and Bob agree to align
their polarizers in either the vertical/horizontal ($\times$) basis, or the
complementary basis of +45/-45 ($+$). Alice then transmits a set of photons
down the quantum channel, randomly choosing one of the two bases for each
photon, whilst Bob measures in his choice of basis for each photon received.

\begin{center}
\begin{tabular}{c c c}
	$\ket{H}$  & codes for & $0_+$ \\
	$\ket{V}$  & codes for & $1_+$ \\
	$\ket{+45}$  & codes for & $0_\times$ \\
	$\ket{-45}$  & codes for & $1_\times$ \\
\end{tabular}
\end{center}

Both Alice and Bob now have a list of sent of received bits, each with a basis. The second
phase of the protocol is the classical sifting step: over the unsecured classical channel, they
first start by comparing which bases were used for each bit, and discarding the bits for which
they used different bases. For a list of size $N$ this step will on average reduce its size by half,
to $N/2$. Having matched the measurement basis, this list is referred to as the \textit{raw key}.

Alice and Bob now reveal a random sample of their raw key to one another, to compare for errors. By
comparing each bit publicly over the classical channel, they can estimate the error rate for
the quantum channel. If no errors are found, the raw key is already the secret key. If there are
errors, Alice and Bob must either correct for them or discard their key INSERT REFERENCE. Both these actions can
take place over the classical channel; hence this step is called the \textit{classical post-processing}.
At the end of this phase, depending on how much information Eve could possess, Alice and Bob either share
a secret key or they must discard the potentially compromised key.

\subsection{B92}
The B92 protocol, involving just 2 non-orthogonal states, is the most minimalist QKD protocol in terms
of encoding. Described by Bennet in 1992, %TODO insert reference%
the B92 coding allows the receiver to learn
whenever they get the bit sent without further discussion from Alice. Although sometimes easier to implement
than the more popular BB84, it is more difficult to establish unconditional security in the case of B92,
as it is much more sensitive to noise in the quantum channel.

\subsubsection{The protocol}
B92 can be carried out using the two bases used in BB84. Following on from the above description, Alice sends either
$\ket{H}$ or $\ket{+45}$, and Bob chooses to measure the incoming bit in either the ($\times$) or the ($+$)
basis. Both of these states code for 0 in BB84; if Bob measures the state that codes for 1 ($\ket{-45}$ or
$\ket{V}$ respectively) he can be certain that Alice transmitted her bit in the opposite basis. Hence, B92
allows the participants to restrict their transmissions in the sifting phase to whether or not Bob made
a conclusive measurement for each bit sent. After discarding all bits for which Bob cannot be certain, the raw
key they are left with is on average $N/4$ bits long, for an initial transmission of $N$ bits.

After a simple comparison of raw key length to the length of the initial string sent by Alice for the classical
post processing step, the participants either possess a secret key, or have been alerted to the possibility
of Eve's presence.

\section{Limitations of QKD - Eavesdropping techniques}
In most literature on QKD, Eve is assumed for the completeness of security proofs to have no restrictions
on her eavesdropping, other than the fundamental limitations of quantum mechanics itself; namely, the
no-cloning theorem and the effects of measurement on a free particle. Briefly described here are the
most commonly considered eavesdropping techniques that are considered for BB84 and B92.

\subsubsection{Intecept-resend}
Potentially the most straightforward eavesdropping technique generally considered is the intercept-resend
strategy, which functions exactly the same for both BB84 and  B92.
Eve essentially takes the same actions as Bob; with her unrestricted capabilities, performs
a quantum non-demolition measurement on the incoming photons from Alice, measuring in either the $\times$
or the $+$ basis. If she has measured in the basis used by Alice, the photon continues on to Bob,
and Eve has obtained full information whilst introducing no errors to the signal. However, if she has measured
in the wrong basis, her result will be uncorrelated with Alice's; meanwhile, she will have sent along
a modified state, so even if Bob measures in the same basis as sent by Alice, half the time he will get the
wrong result. %TODO insert ref review paper%

Intercept-resend leaves Eve with full information of half the bits in the full key ($I_E = 0.5$). It introduces
an error rate of 25\% ($Q = 0.25$) into the key recieved by Bob. Using the assumptions given by (Csiszar and Korner, 1978) %TODO: properly reference
it can be shown that this means no secret key can be extracted. This is in fact true for all protocols.

Even in the case where Eve does not eavesdrop on all photons in the key, and instead on just a fraction $p$,
then clearly $Q = p/4$ and so

\begin{align}
I_E	= p/2 = 2Q
\end{align}

Hence a secure key can only be extracted if $Q \geq 17\%$. The precise figure varies by%TODO ref PERFORMANCE COMPARISON OF BB84 AND B92 SATELLITE-BASED FREE SPACE QUANTUM OPTICAL COMMUNICATION SYSTEMS IN THE PRESENCE OF CHANNEL EFFECTS
publication however. %TODO GIVE EXAMPLE

\subsection{Photon Number Splitting attack - PNS} %TODO: use Robustness Kronberg paper for reference for this
In real world applications of QKD, the photon source is never a true strong single photon emitter. Typically
a weak coherent pulse laser source is used. %TODO: reference
These sources emit pulses of coherent states, often containing 2 or more photons.

This presents a problem for the security of BB84 and B92, as described below.

\subsubsection{BB84}

The security weakness in coherent pulses comes from Eve's ability to nondestructively determine the number of photons
in a pulse. The principle of the PNS attack is that Eve will block all single photon pulses, and for any pulse where
multiple photons are detected, Eve will store in quantum memory a subset of the pulse, while allowing the rest to continue
on to Bob (through an ideal quantum channel, to ensure his receipt of the photons). During the sifting
phase, Alice and Bob's communications over the public channel can be used by Eve to construct
a complete key from the stored states. The major benefit to Eve with this method is that it does not introduce an error in the
states received by Bob; in fact, the only risk of detection comes from the losses due to single photon pulses being blocked.

As long as these losses do not reach a critical threshold (dependent on the length of the communication channel and the expected losses),
Eve's actions remain undetected, and the security of the protocol is compromised.


\subsubsection{B92}

Against B92, the PNS attack is even more straightforward. No quantum memory is required; all that is sufficient is that Eve performs
the same measurement as Bob, and blocks the message in the event of an inconclusive result. In this way, the losses are attributed
to channel losses; as long as a critical threshold is not reached (as above), Eve will obtain full information of the key. Again, due to
the lack of error in the received states, Eve will remain entirely unnoticed.



%------------------------------------------------


%----------------------------------------------------------------------------------------

\end{document}
